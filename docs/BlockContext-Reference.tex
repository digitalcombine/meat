\subsection {BlockContext Class}

\begin{displaymath}
  \xymatrix @-1pc {
    & \txt{Origin Context} \ar [dl] \ar@{<=>}[dd] \\
    \txt{Messaging Context} \ar@{<->}[dr] & \\
    & \txt{Block Context}\\
  }
\end{displaymath}

The origin context is the context in which the block context byte code was found and the block context was created. The local variables for the origin context are prepended to the block context local variables.

The messaging context is the context where one of the \textit{execute} method is messaged.

\begin{center}
  \begin{tabular}{ c | l }
  	Variable   & Context        \\ \hline
  	self       & Origin Context \\
  	class      & Origin Context \\
  	context    & Block Context  \\
  	null       & Origin Context \\
  	4 \dots    & Origin Context \\
  	\dots last & Block Context
  \end{tabular}
\end{center}


\begin{lstlisting}
Context subClass: BlockContext body: {
  method execute
  method executeOnBreak: block
  method executeOnBreak: breakBlock onContinue: continueBlock
  method executeOnContinue: block
  method localVariables
  method getLocal: index
  method setLocal: index to: value
  method return
  method return: result
  method continue
  method break
}
\end{lstlisting}

\subsubsection{method execute}
Executes the block. After execution the block must be reset to
allow the block to be executed again.

\subsubsection{method executeOnBreak: block}

\subsubsection{method executeOnBreak: breakBlock onContinue: continueBlock}

\subsubsection{method executeOnContinue: block}

\subsubsection{method localVariables}

\subsubsection{method getLocal: index}

\subsubsection{method setLocal: index to: value}

\subsubsection{method return}
Stops execution of this context and messages the \textit{return} method of the origin context.

\subsubsection{method return: result}
Stops execution of this context and messages the \textit{return: value} method of the origin context. The \textit{result} is not set to the block context.

\subsubsection{continue}

\subsubsection{break}
If the Block Context was messaged with \textit{executeOnBreak:} or \textit{executeOnBreak:onContinue:} then the Block Context ends execution and returns to the Messaging Context to execute appropriate break block.

If the Block Context was messaged with \textit{execute} or \textit{executeonContinue:} then \textit{break} is messaged to the origin context.
