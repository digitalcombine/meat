\subsection {Object Class}

The Object class is the base class for all objects and classes.

\begin{lstlisting}
Object subclass: Object body: {
  class method == other
  class method <> other
  class method is: other
  class method isNot: other
  class method isObject
  class method isClass
  class method isNull
  class method type
  class method isType: class
  class method weakReference
  class method normalReference
  class method isWeakReference
  method initialize
  method cleanup
  method == other
  method <> other
  method is: other
  method isNot: other
  method isObject
  method isClass
  method isNull
  method type
  method isType: class
  method weakReference
  method normalReference
  method isWeakReference
}
\end{lstlisting}

\subsubsection {The == methods}
\begin{lstlisting}
  class method == other
  method == other
\end{lstlisting}

Tests if the \textit{other} object or class is the same object as self. For the Object class, it does not test the equality of values, but sub-classes should override this method to test for value equality. The result is to be either the True or False Boolean objects.

\hfill
\subsubsection {The $<>$ methods}
\begin{lstlisting}
  class method <> other
  method <> other
\end{lstlisting}

Tests if the \textit{other} object or class is not the same object as self. For this class, it does not test the inequality of values, but sub-classes should override this method to test for value inequality. The result is to be either the True or False Boolean objects.

\hfill
\subsubsection {The is: methods}
\begin{lstlisting}
  class method is: other
  method is: other
\end{lstlisting}

Tests if the \textit{other} object or class is the same object as self. This method should not be overridden by subclasses. The result is to be either the True or False Boolean objects.

\hfill
\subsubsection {The isNot: methods}
\begin{lstlisting}
  class method isNot: other
  method isNot: other
\end{lstlisting}

Tests if the \textit{other} object or class is not the same object as self. This method should not be overridden by subclasses. The result is to be either the True or False Boolean objects.

\hfill
\subsubsection {The isObject methods}
\begin{lstlisting}
  class method isObject
  method isObject
\end{lstlisting}

Tests if the object is an object. The result is to be either the True or False Boolean objects.

The class method always returns the False object and the object method always returns the True object.

\hfill
\subsubsection {The isClass methods}
\begin{lstlisting}
  class method isClass
  method isClass
\end{lstlisting}

Tests if the object is a class. The result is to be either the True or False Boolean objects.

The class method always returns the True object and the object method always returns the False object.

\hfill
\subsubsection {The isNull methods}
\begin{lstlisting}
  class method isNull
  method isNull
\end{lstlisting}

Tests if the object or class is the Null object. These methods always return the False object and should only be overridden by the Null class.

\hfill
\subsubsection {The type methods}
\begin{lstlisting}
  class method type
  method type
\end{lstlisting}

Returns the class for the object. The class method always returns the class Class.

\hfill
\subsubsection {The isType methods}
\begin{lstlisting}
class method isType: type
method isType: type
\end{lstlisting}

\hfill
\subsubsection {The weakReference methods}
\begin{lstlisting}
  class method weakReference 
  method weakReference
\end{lstlisting}

Returns a reference to the object that does not affect the normal memory management reference counting.

\hfill
\subsubsection {The normalReference methods}
\begin{lstlisting}
  class method normalReference
  method normalReference
\end{lstlisting}

Returns a reference to the object that does the normal memory management reference counting.

\hfill
\subsubsection {The isWeakReference methods}
\begin{lstlisting}
class method isWeakReference
method isWeakReference
\end{lstlisting}

Returns the true object if the object reference is a weak reference otherwise the false object is returned.

\hfill
\subsubsection {method initialize}
When a class creates a new object
(\hyperref[sec:class_method_newobject]{see newObject method}) this method is messaged to populate the objects properties with default values. By default, the properties of a new object are all initially set to the Null object. This method can be overridden by sub-classes to set it's properties to something other than null and must always message the super class initialize method.

\begin{lstlisting}
  method initialize {
    super initialize
    # Populate object's properties here.
  }
\end{lstlisting}

\subsubsection {method cleanup}
