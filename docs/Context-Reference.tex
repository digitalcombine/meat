\subsection {Context Class}

A context is the execution context when a method answers a message.
\begin{center}
  \begin{tabular}{ c | l }
    Local Variable & Value\\ \hline
    0 & self \\
    1 & class \\
    2 & context (weak reference) \\
    3 & null \\
    4 \dots & message parameters \\
    \dots last & local variables \\
  \end{tabular}
\end{center}

\begin{lstlisting}
Object subClass: Context body: {
  class method subClass: name body: definition
  class method newObject
  method messenger
  method localVariables
  method getLocal: index
  method setLocal: index to: value
  method repeat: block
  method return
  method return: result
  method break
  method continue
}
\end{lstlisting}

\subsubsection {class method subClass: name body: definition}
The Context class should never be sub-classed therefore this method always
raises an exception.

\subsubsection {class method newObject}
Context classes are only created internally when an object is messaged
therefore this method always raises an exception.

\subsubsection {method messenger}
This method returns the context object where the message was
initiated. The null object can be returned if it was the very first context
created to start execution.

\subsubsection {method localVariables}
This method returns the number of local variables allocated in the context
minus \textit{self}, \textit{class}, \textit{context} and \textit{null}.

\subsubsection {method getLocal: index}

\subsubsection {method setLocal: index to: value}

\subsubsection {method repeat: block}
Repeat the execution of \textit{block} infinitely. \textit{block} must be
a BlockContext object. Any other object type will raise an Exception.

\subsubsection {method return}

\subsubsection {method return: result}

\subsubsection{method break}
This method does nothing.

\subsubsection{method continue}
