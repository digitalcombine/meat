\documentclass [10pt]{article}
\usepackage[margin=2cm]{geometry}
\usepackage[colorlinks=true,linkcolor=blue,urlcolor=blue]{hyperref}
\usepackage{listings}
\usepackage{graphicx}
\usepackage{tabulary}
\usepackage{longtable}
\usepackage{multirow}
\usepackage{hyperref}
\usepackage{mathtools}
\usepackage[all]{xy}

\title {Meat Language Reference Standard}
\author {Ron R Wills \textless\href{mailto:ron.rwsoft@gmail.com}{ron.rwsoft@gmail.com}\textgreater}

\begin {document}
\pagenumbering {gobble}
\begin{titlepage}
	\centering
	\includegraphics[width=0.35\textwidth]{meat-logo.jpg}\par\vspace{2cm}
	{\scshape\Large Language Version 1.0 Proposal\par}
	\vspace{1.5cm}
	{\huge\bfseries Meat Language Standard Reference\par}
	\vspace{2cm}
	{\Large\itshape Ron R Wills \texttt{\textless\href{mailto:ron.rwsoft@gmail.com}{ron.rwsoft@gmail.com}\textgreater}}\par

	\vfill

% Bottom of the page
	{\large \today\par}
\end{titlepage}

\noindent{}Copyright \copyright{} 2017 Ron R Wills \texttt{\textless\href{mailto:ron.rwsoft@gmail.com}{ron.rwsoft@gmail.com}\textgreater}.\\[.5 cm]
Permission is granted to copy, distribute and/or modify this document
under the terms of the GNU Free Documentation License, Version 1.3
or any later version published by the Free Software Foundation;
with no Invariant Sections, no Front-Cover Texts, and no Back-Cover Texts.
A copy of the license is included in the section entitled \textit{"GNU
Free Documentation License"}.
\tableofcontents
\newpage
\pagenumbering {arabic}

\section {Introduction}
\subsection {Scope}
The purpose of this document is not to be a learning guide to the language but
to define the Meat language itself.

\newpage
\section {Core Language Components}

All the components of the language described in this section are considered
mandatory to the language.

\subsection {Byte Interpretation}

\subsubsection{Integer Values}

\begin{center}
  \begin{tabular}{ r | c r r }
    & Size (bits) & Original Value & Endian Swapped Value\\ \hline
    Byte & 8 & $12_{16}$ & $12_{16}$ \\
    Word & 16 & $1234_{16}$ & $3412_{16}$ \\
    Integer & 32 & $12345678_{16}$ & $56781234_{16}$ \\
    Long & 64 & $\mathrm{123456789ABCDEF0}_{16}$ & \\
  \end{tabular}
\end{center}

\subsection{Virtual Machine Bytecode}

\begin{center}
  \begin{tabular}{ | p{6cm} | c | p{8cm} | }
    \hline
    Operation & Code & Brief Description \\ \hline
    No Operation & $00_{16}$ & Only increament the bytecode pointer by one.
                              Nothing else is done. \\
    Message Object & $01_{16}$ & Send a message to an object. \\
    Message Object with Result & $02_{16}$ & Send a message to the object
                                            and assigning the result to a
                                            local variable. \\
    Message Object's Super Class & $03_{16}$ & Nothing is \\
    Message Object's Super Class with Result & $04_{16}$ & Nothing is \\
    Beginning of Block Context & $\mathrm{0A}_{16}$ & Nothing is \\
    End of Context & $\mathrm{0B}_{16}$ & Nothing is \\
    Assign Local Variable & $10_{16}$ & ...\\
    Assign Object Property & $11_{16}$ & ...\\
    Assign Class Property & $12_{16}$ & ...\\
    Assign Class to Local Variable & $13_{16}$ & ...\\
    Assign an Integer Constant to a Local Variable & $14_{16}$ & ...\\
    \hline
  \end{tabular}
\end{center}

\subsection{Library File Format}

\begin{center}
  \begin{tabular}{ |l|l|p{8cm}| }
    \hline
    \multirow{5}{*}{Header} & \texttt{unsigned bytes [4]} = ''MLIB'' & The
    magic file identifier\\ \cline{2-3}
    & \texttt{unsigned byte} = 1 & Major Meat library format version\\
    \cline{2-3}
    & \texttt{unsigned byte} = 0 & Minor Meat library format version\\
    \cline{2-3}
    & \texttt{unsigned word} = 0 & Reserved for future flags.\\ \cline{2-3}
    & \texttt{unsigned integer} = 0 & A hash ID for an application class for
    the virtual interpreter to message if the library is being executed.
    If this field is set to zero, then the library is not considered an
    executable application, but can still be loaded as a library.

    The hash ID must refer to an Application class within the same library.
    Any other reference is considered invalid and an Exception must be
    thrown to indicate the error by the virtual interpreter.\\
    \hline

    \multirow{2}{*}{Library Imports} & \texttt{unsigned byte} & The total number of libraries that need to be imported.\\ \cline{2-3}
    & \texttt{unsigned byte arrays} & A list of ''0'' terminated strings of each library name to be imported.\\ \hline

    \multirow{2}{*}{Serialized Class Objects} & \texttt{unsigned byte} & The total number of Classes serialized in the library.\\ \cline{2-3}
    & \texttt{serialized class array} & An array of all the serialized classes. see \hyperref[sec:class_serial]{Class Serialization}.\\ \hline
  \end{tabular}
\end{center}

\subsubsection{Class Serialization}
\label{sec:class_serial}

\textbf{Serialized Class Format}
\begin{center}
  \begin{tabular}{|l|l|p{8cm}|}
    \hline
    \multirow{3}{*}{Header} & \texttt{unsigned integer} & The hashed ID of the class name.\\ \cline{2-3}
    & \texttt{unsigned byte} & The number of class properties.\\ \cline{2-3}
    & \texttt{unsigned byte} & The number of object properties.\\ \hline

    \multirow{2}{*}{Virtual Table} & \texttt{unsigned byte} & The number of virtual table entries.\\ \cline{2-3}
    & \texttt{unsigned byte} & The number of class virtual table entries.\\ \hline

    \multirow{2}{*}{Byte Code} & \texttt{unsigned word} & Size of the byte code array.\\ \cline{2-3}
    & \texttt{unsigned byte array} & The byte code.\\ \hline
  \end{tabular}
\end{center}

\textbf{Virtual Table Entries}
\begin{center}
  \begin{tabular}{|l|p{8cm}|}
    \hline
    \texttt{unsigned integer} & Hashed ID of the method name.\\ \hline
    \texttt{unsigned integer} & Hashed ID of the class where the method is found.\\ \hline
    \texttt{unsigned byte} & Entry flags.\\ \hline
    \texttt{unsigned byte} & The number of local variables required during execution.\\ \hline
    \texttt{unsigned word} & Byte code offset to the start of the method.\\ \hline
  \end{tabular}
\end{center}

\textbf{Virtual Table Entry Flags}
\begin{displaymath}
  \xymatrix @-1pc {Bit &*+[F]{7} \ar @{-} [ddd]&*+[F]{6} \ar @{-} [dd] &*+[F]{5} \ar @{-} [dd] &*+[F]{4} \ar @{-} [dd] &*+[F]{3} \ar @{-} [dd] &*+[F]{2} \ar @{-} [dd]
    &*+[F]{1} \ar @{-} [dd] &*+[F]{0} \ar @{-} [d] & \\
    & & & & & & & & \ar [r] & \txt<6cm>{If set the virtual table entry is an offset into the byte code, otherwise it is an internal function pointer.}\\
    & & \ar [rrrrrrr] & & & & & & & \txt<6cm>{Reserved for future use.}\\
    & \ar [rrrrrrrr] & & & & & & & & \txt<6cm>{Method found in a super class.}}
\end{displaymath}


\newpage
\section {Core Language Builtin Library}
\subsection {Object Class}

The Object class is the base class for all objects and classes.

\begin{lstlisting}
Object subclass: Object body: {
  class method == other
  class method <> other
  class method is: other
  class method isNot: other
  class method isObject
  class method isClass
  class method isNull
  class method type
  class method isType: class
  class method weakReference
  class method normalReference
  class method isWeakReference
  method initialize
  method cleanup
  method == other
  method <> other
  method is: other
  method isNot: other
  method isObject
  method isClass
  method isNull
  method type
  method isType: class
  method weakReference
  method normalReference
  method isWeakReference
}
\end{lstlisting}

\subsubsection {The == methods}
\begin{lstlisting}
  class method == other
  method == other
\end{lstlisting}

Tests if the \textit{other} object or class is the same object as self. For the Object class, it does not test the equality of values, but sub-classes should override this method to test for value equality. The result is to be either the True or False Boolean objects.

\hfill
\subsubsection {The $<>$ methods}
\begin{lstlisting}
  class method <> other
  method <> other
\end{lstlisting}

Tests if the \textit{other} object or class is not the same object as self. For this class, it does not test the inequality of values, but sub-classes should override this method to test for value inequality. The result is to be either the True or False Boolean objects.

\hfill
\subsubsection {The is: methods}
\begin{lstlisting}
  class method is: other
  method is: other
\end{lstlisting}

Tests if the \textit{other} object or class is the same object as self. This method should not be overridden by subclasses. The result is to be either the True or False Boolean objects.

\hfill
\subsubsection {The isNot: methods}
\begin{lstlisting}
  class method isNot: other
  method isNot: other
\end{lstlisting}

Tests if the \textit{other} object or class is not the same object as self. This method should not be overridden by subclasses. The result is to be either the True or False Boolean objects.

\hfill
\subsubsection {The isObject methods}
\begin{lstlisting}
  class method isObject
  method isObject
\end{lstlisting}

Tests if the object is an object. The result is to be either the True or False Boolean objects.

The class method always returns the False object and the object method always returns the True object.

\hfill
\subsubsection {The isClass methods}
\begin{lstlisting}
  class method isClass
  method isClass
\end{lstlisting}

Tests if the object is a class. The result is to be either the True or False Boolean objects.

The class method always returns the True object and the object method always returns the False object.

\hfill
\subsubsection {The isNull methods}
\begin{lstlisting}
  class method isNull
  method isNull
\end{lstlisting}

Tests if the object or class is the Null object. These methods always return the False object and should only be overridden by the Null class.

\hfill
\subsubsection {The type methods}
\begin{lstlisting}
  class method type
  method type
\end{lstlisting}

Returns the class for the object. The class method always returns the class Class.

\hfill
\subsubsection {The isType methods}
\begin{lstlisting}
class method isType: type
method isType: type
\end{lstlisting}

\hfill
\subsubsection {The weakReference methods}
\begin{lstlisting}
  class method weakReference 
  method weakReference
\end{lstlisting}

Returns a reference to the object that does not affect the normal memory management reference counting.

\hfill
\subsubsection {The normalReference methods}
\begin{lstlisting}
  class method normalReference
  method normalReference
\end{lstlisting}

Returns a reference to the object that does the normal memory management reference counting.

\hfill
\subsubsection {The isWeakReference methods}
\begin{lstlisting}
class method isWeakReference
method isWeakReference
\end{lstlisting}

Returns the true object if the object reference is a weak reference otherwise the false object is returned.

\hfill
\subsubsection {method initialize}
When a class creates a new object
(\hyperref[sec:class_method_newobject]{see newObject method}) this method is messaged to populate the objects properties with default values. By default, the properties of a new object are all initially set to the Null object. This method can be overridden by sub-classes to set it's properties to something other than null and must always message the super class initialize method.

\begin{lstlisting}
  method initialize {
    super initialize
    # Populate object's properties here.
  }
\end{lstlisting}

\subsubsection {method cleanup}

\subsection{Class Class}

The class Class is the base type for all Class objects.

\begin{lstlisting}
Object subClass: Class body: {
  class method subClass: name body: definition
  class method newObject
  class method superClass
}
\end{lstlisting}

\subsubsection{class method subClass: name body: definition}
This method should raise an Exception. The functionality for this method
is to be provided by a compiler library. How a compiler library hooks in the
functionaliy is up to the underlying implementation and is not defined here.

\subsubsection{class method newObject}
\label{sec:class_method_newobject}
Method used for creating constructor class methods. This method creates a new
object and sets the self object to it then messages \textit{self initialize}.

All constructor methods should follow the following template:

\begin{lstlisting}
class method new {
  self newObject
  # Add any additional object initialization here.
  context return: self
}
\end{lstlisting}

\subsubsection{class method superClass}
Returns the super class for the Class.

\subsection {Context Class}

A context is the execution context when a method answers a message.
\begin{center}
  \begin{tabular}{ c | l }
    Local Variable & Value\\ \hline
    0 & self \\
    1 & class \\
    2 & context (weak reference) \\
    3 & null \\
    4 \dots & message parameters \\
    \dots last & local variables \\
  \end{tabular}
\end{center}

\begin{lstlisting}
Object subClass: Context body: {
  class method subClass: name body: definition
  class method newObject
  method messenger
  method localVariables
  method getLocal: index
  method setLocal: index to: value
  method return
  method return: result
}
\end{lstlisting}

\subsubsection {class method subClass: name body: definition}
The Context class should never be sub-classed therefore this method always
raises an exception.

\hfill
\subsubsection {class method newObject}
Context classes are only created internally when an object is messaged
therefore this method always raises an exception.

\hfill
\subsubsection {method messenger}
This method returns the context object where the message was
initiated. The null object can be returned if it was the very first context
created to start execution.

\subsection {BlockContext Class}

\begin{displaymath}
  \xymatrix @-1pc {
    \txt{Initial Context} \ar [r] & \txt{Block Context} \\
    \txt{Execution Context} \ar [ur] & \\
  }
\end{displaymath}

\begin{lstlisting}
Context subClass: BlockContext body: {
  method execute
  method localVariables
  method getLocal: index
  method setLocal: index to: value
  method return
  method return: result
}
\end{lstlisting}

\subsection {Null Class and Object}

The Null object is to indicate that a reference doesn't actually reference
any object. The Null class has only one object to define null and can not
be subclassed.

\begin{lstlisting}
Object subclass: Null body: {
  class method newObject
  class method subClass: name body: definition
  class method object
  method isNull
}
\end{lstlisting}

\subsubsection{Class method newObject}
Always raises an Exception.

\subsubsection{Class method subClass: body:}
Always raises an Exception.

\subsubsection{Class method object}
The method returns the null object.

\subsubsection{Method isNull}
The method returns the true object.

\subsection {Exception Class}

\begin{lstlisting}
Object subclass: Exception body: {
  class method try: block
  class method try: block catch: catchBlock
  class method try: block catch: .error do: catchBlock
  class method throw
  class method throw: message
  class method throwFor: context
  class method throw: message for: context
  method message
  method context
}
\end{lstlisting}

\subsubsection{Class method try:}

\subsection {Boolean Class and Objects}

\begin{lstlisting}
Object subClass: Boolean body: {
  class method true
  class method false
  method isTrue: block
  method isTrue: trueBlock else: falseBlock
  method isFalse: block
  method isFalse: falseBlock else: trueBlock
  method not
  method and: other
  method or: other
  method xor: other
  method asText
}
\end{lstlisting}

\subsection{Numeric Class}

The Numeric Class is an abstract base class for all numeric or mathematical values.

\begin{lstlisting}
Object subClass: Numeric body: {
  class method minValue
  class method maxValue
  method + other
  method - other
  method * other
  method / other
  method % other
  method ^ other
  method == other
  method != other
  method < other
  method <= other
  method > other
  method >= other
  method negative
  method absolute
  method compliment
  method asText
}
\end{lstlisting}
\subsection{Integer Class}

\begin{lstlisting}
Numeric subClass: Integer body: {
  class method minValue
  class method maxValue
  method + other
  method - other
  method * other
  method / other
  method % other
  method ^ other
  method == other
  method != other
  method < other
  method <= other
  method > other
  method >= other
  method negative
  method absolute
  method sign
  method or: other
  method and: other
  method xor: other
  method rshift
  method rshift: amount
  method lshift
  method lshift: amount
  mothod timesDo: block
  method asText
}
\end{lstlisting}

\subsection{Number Class}

\begin{lstlisting}
Numeric subClass: Number body: {
  class method cos: angle
  
}
\end{lstlisting}
\subsection {Text Class}

\begin{lstlisting}
Object subclass: Text body: {
  method isEmpty
  method length
  method getCharAt: index
  method getFrom: start count: end
  method replaceFrom: start count: end with: value
  method replaceAll: text with: value
  method findFirst: other
  method findFirst: other at: index
  method findLast: other
  method findLast: other at: index
  method forEach: .character do: block
  method asLowercase
  method asUppercase
  method copy
  method asNumber
  method asInteger
  method asText
  method == other
  method <> other
  method < other
  method <= other
  method > other
  method >= other
  method + other
  method * amount
}
\end{lstlisting}

\subsubsection{method isEmpty}
Returns the true Boolean object if the length of the Text object is equal to zero otherwise the false Boolean object is returned.

\subsubsection{method length}
Returns an Integer object with the number of Unicode characters in the text.

\subsubsection{method getCharAt: index}
Returns a Text object of the character found at the numeric index. The first character of the text object start at index 1.

If the index is less than 1 or is greater than the length of the text an Exception must be thrown.

\subsubsection{method getFrom: start count: end}

\subsubsection{method replaceFrom: start count: end with: value}

\subsubsection{method replaceAll: text with: value}

\subsubsection{method findFirst: value}

\subsubsection{method findFirst: value at: index}

\subsubsection{method findLast: value}

\subsubsection{method findLast: value at: index}

\subsubsection{forEach: .character do: block}

\subsubsection{asLowercase}

\subsubsection{asUppercase}

\subsubsection{copy}

\subsubsection{asNumber}

\subsubsection{asInteger}

\subsubsection{asText}

\subsubsection{method $==$ other}

\subsubsection{method $<>$ other}

\subsubsection{method $<$ other}

\subsubsection{method $<=$ other}

\subsubsection{method $>$ other}

\subsubsection{method $>=$ other}

\subsubsection{method $+$ other}

\subsubsection{method $*$ amount}

\subsection{List Class}

\begin{lstlisting}
Object subClass: List body: {
  method copy
}
\end{lstlisting}

\subsubsection{method copy}

\subsection {Set Class}

\begin{lstlisting}
Object subclass: Set body: {
  class method new
}
\end{lstlisting}
\subsection {Index Class}

\begin{lstlisting}
Object subclass: Index body: {
  class method new
  method get: key
  method set: key to: value
  method has: key
  method remove: key
  method entries
  method getKeyList
  method getValueList
  method clear
  method isEmpty
  method copy
  method forEach: key with: value do: body
  method forEachKey: key do:
  method forEachValue: value do:
}
\end{lstlisting}

\subsection{Archive Class}

\begin{lstlisting}
Object subclass: Archive body: {

}
\end{lstlisting}
\subsection{Library Class}

\begin{lstlisting}
Object subClass: Library body: {
  class method exists: libraryName
  class method import: libraryName
  class method requires: libraryName
  class method include: nativeCode
  class method setApplication: class
  class method unload: libraryName
  class method listSearchPaths
}
\end{lstlisting}

\subsubsection{class method exists:}
Checks every search path directory for the library given by \textit{libraryName} and return the true Boolean object if the library was found otherwise it returns the false Boolean object.

\subsubsection{class method import:}

\subsubsection{class method requires:}
By default this method always throws an Exception. This method should be hooked when a compiler is loaded.
\input{Application-Reference}

\newpage
\section {Grinder Lexical Elements and Parsing}
\subsection {Characters}
The Unicode UTF-8 character set

The Unicode white space characters \texttt{U+0009}
\textit{(the TAB character \textbackslash t)}, \texttt{U+0020}
\textit{(the SPACE character)} and
\texttt{U+00A0} \textit{(the NO-BREAK SPACE character)} are used as delimiters
between the token components of a command. These characters can be used
multiple times together to form a single delimiter.

A line termination or a \textit{newline} is definied by the character set
described in the following table. Code parsers must accept all of these
characters as a line terminator.

\begin{center}
  \begin{tabular}{|l|l|l|}
  	\hline
  	                          & \textbf{Unicode}                & \textbf{Escaped Character}        \\ \hline
  	\textbf{LF}               & \texttt{U+000A}                 & \textbackslash n                  \\ \hline
  	\textbf{VT}               & \texttt{U+000B}                 & \textbackslash v                  \\ \hline
  	\textbf{FF}               & \texttt{U+000C}                 & \textbackslash f                  \\ \hline
  	\textbf{CR}               & \texttt{U+000D}                 & \textbackslash r                  \\ \hline
  	\textbf{CR} + \textbf{LF} & \texttt{U+000D} \texttt{U+000A} & \textbackslash r \textbackslash n \\ \hline
  	\textbf{LS}               & \texttt{U+2028}                 &  \\ \hline
  	\textbf{PS}               & \texttt{U+2029}                 &  \\ \hline
  \end{tabular}
\end{center}

\subsubsection {Numeric Chararcters}

Using the unicode character set allows code to be in any language supported.
This also extends to numeric character sets. The following table describes the
numeric characters that meat parsers must recognize as numeric values and
translate into their binary equivalent.

\begin{center}
  \begin{longtable}{|l|c|}
  	\hline
  	                                             &         \textbf{Unicode}          \\ \hline
  	\textbf{Digits}                              &  \texttt{U+0030}-\texttt{U+0039}  \\ \hline
  	\textbf{Arabic-Indic Digits}                 &  \texttt{U+0660}-\texttt{U+0669}  \\ \hline
  	\textbf{Extended Arabic-Indic Digit}         &  \texttt{U+06F0}-\texttt{U+06F9}  \\ \hline
  	\textbf{NKo Digits}                          &  \texttt{U+07C0}-\texttt{U+07C9}  \\ \hline
  	\textbf{Devavagari Digits}                   &  \texttt{U+0966}-\texttt{U+096F}  \\ \hline
  	\textbf{Bengali Digits}                      &  \texttt{U+09E6}-\texttt{U+09EF}  \\ \hline
  	\textbf{Gurmukhi Digits}                     &  \texttt{U+0A66}-\texttt{U+0A6F}  \\ \hline
  	\textbf{Gujarati Digits}                     &  \texttt{U+0AE6}-\texttt{U+0AEF}  \\ \hline
  	\textbf{Oriya Digits}                        &  \texttt{U+0B66}-\texttt{U+0B6F}  \\ \hline
  	\textbf{Tamil Digits}                        &  \texttt{U+0BE6}-\texttt{U+0BEF}  \\ \hline
  	\textbf{Telugu Digits}                       &  \texttt{U+0C66}-\texttt{U+0C6F}  \\ \hline
  	\textbf{Kannada Digits}                      &  \texttt{U+0CE6}-\texttt{U+0CEF}  \\ \hline
  	\textbf{Malayalam Digits}                    &  \texttt{U+0D66}-\texttt{U+0D6F}  \\ \hline
  	\textbf{Sinhala Digits}                      &  \texttt{U+0DE6}-\texttt{U+0DEF}  \\ \hline
  	\textbf{Thai Digits}                         &  \texttt{U+0E50}-\texttt{U+0E59}  \\ \hline
  	\textbf{Lao Digits}                          &  \texttt{U+0ED0}-\texttt{U+0ED9}  \\ \hline
  	\textbf{Tibetan Digits}                      &  \texttt{U+0F20}-\texttt{U+0F29}  \\ \hline
  	\textbf{Myanmar Digits}                      &  \texttt{U+1040}-\texttt{U+1049}  \\ \hline
  	\textbf{Myanmar Shan Digits}                 &  \texttt{U+1090}-\texttt{U+1099}  \\ \hline
  	\textbf{Khmer Digits}                        &  \texttt{U+17E0}-\texttt{U+17E9}  \\ \hline
  	\textbf{Mongolian Digits}                    &  \texttt{U+1810}-\texttt{U+1819}  \\ \hline
  	\textbf{Limbu Digits}                        &  \texttt{U+1946}-\texttt{U+194F}  \\ \hline
  	\textbf{New Tai Lue Digits}                  &  \texttt{U+19D0}-\texttt{U+19D9}  \\ \hline
  	\textbf{Tai Tham Hora Digits}                &  \texttt{U+1A80}-\texttt{U+1A89}  \\ \hline
  	\textbf{Tai Tham Tham Digits}                &  \texttt{U+1A90}-\texttt{U+1A99}  \\ \hline
  	\textbf{Balinese Digits}                     &  \texttt{U+1B50}-\texttt{U+1B59}  \\ \hline
  	\textbf{Sundanese Digits}                    &  \texttt{U+1BB0}-\texttt{U+1BB9}  \\ \hline
  	\textbf{Lepcha Digits}                       &  \texttt{U+1C40}-\texttt{U+1C49}  \\ \hline
  	\textbf{Ol Chiki Digits}                     &  \texttt{U+1C50}-\texttt{U+1C59}  \\ \hline
  	\textbf{Vai Digits}                          &  \texttt{U+A620}-\texttt{U+A629}  \\ \hline
  	\textbf{Saurashtra Digits}                   &  \texttt{U+A8D0}-\texttt{U+A8D9}  \\ \hline
  	\textbf{Kayah Li Digits}                     &  \texttt{U+A900}-\texttt{U+A909}  \\ \hline
  	\textbf{Javanese Digits}                     &  \texttt{U+A9D0}-\texttt{U+A9D9}  \\ \hline
  	\textbf{Myanmar Tai Laing Digits}            &  \texttt{U+A9F0}-\texttt{U+A9F9}  \\ \hline
  	\textbf{Cham Digits}                         &  \texttt{U+AA50}-\texttt{U+AA59}  \\ \hline
  	\textbf{Meetei Mayek Digits}                 &  \texttt{U+ABF0}-\texttt{U+ABF9}  \\ \hline
  	\textbf{Fullwidth Digits}                    &  \texttt{U+FF10}-\texttt{U+FF19}  \\ \hline
  	\textbf{Osmanya Digits}                      & \texttt{U+104A0}-\texttt{U+104A9} \\ \hline
  	\textbf{Brahmi Digits}                       & \texttt{U+11066}-\texttt{U+1106F} \\ \hline
  	\textbf{Sora Sompeng Digits}                 & \texttt{U+110F0}-\texttt{U+110F9} \\ \hline
  	\textbf{Chakma Digits}                       & \texttt{U+11136}-\texttt{U+1113F} \\ \hline
  	\textbf{Sharada Digits}                      & \texttt{U+111D0}-\texttt{U+111D9} \\ \hline
  	\textbf{Khudawadi Digits}                    & \texttt{U+112F0}-\texttt{U+112F9} \\ \hline
  	\textbf{Newa Digits}                         & \texttt{U+11450}-\texttt{U+11459} \\ \hline
  	\textbf{Tirhuta Digits}                      & \texttt{U+114D0}-\texttt{U+114D9} \\ \hline
  	\textbf{Modi Digits}                         & \texttt{U+11650}-\texttt{U+11659} \\ \hline
  	\textbf{Takri Digits}                        & \texttt{U+116C0}-\texttt{U+116C9} \\ \hline
  	\textbf{Ahom Digits}                         & \texttt{U+11730}-\texttt{U+11739} \\ \hline
  	\textbf{Warang Citi Digits}                  & \texttt{U+118E0}-\texttt{U+118E9} \\ \hline
  	\textbf{Bhaiksuki Digits}                    & \texttt{U+11C50}-\texttt{U+11C59} \\ \hline
  	\textbf{Masaram Digits}                      & \texttt{U+11D50}-\texttt{U+11D59} \\ \hline
  	\textbf{Mro Digits}                          & \texttt{U+16A60}-\texttt{U+16A69} \\ \hline
  	\textbf{Pahawh Hmong Digits}                 & \texttt{U+16B50}-\texttt{U+16B59} \\ \hline
  	\textbf{Mathematical Bold Digits}            & \texttt{U+1D7CE}-\texttt{U+1D7D7} \\ \hline
  	\textbf{Mathematical Double-Struck Digits}   & \texttt{U+1D7D8}-\texttt{U+1D7EF} \\ \hline
  	\textbf{Mathematical Sans-Serif Digits}      & \texttt{U+1D7E2}-\texttt{U+1D7EB} \\ \hline
  	\textbf{Mathematical Sans-Serif Bold Digits} & \texttt{U+1D7EC}-\texttt{U+1D7F5} \\ \hline
  	\textbf{Mathematical Monospace Digits}       & \texttt{U+1D7F6}-\texttt{U+1D7FF} \\ \hline
  	\textbf{Adlam Digits}                        & \texttt{U+1E950}-\texttt{U+1E959} \\ \hline
  \end{longtable}
\end{center}

\newpage
\section {Appendix}
\input {FDL}

\end {document}
