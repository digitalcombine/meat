\subsection {Byte Interpretation}

\subsubsection{Integer Endian}

\begin{center}
  \begin{tabular}{ c r r }
    Size (bits) & Original Value & Endian Swapped Value\\ \hline
    8 & $12_{16}$ & $12_{16}$ \\
    16 & $1234_{16}$ & $3412_{16}$ \\
    32 & $12345678_{16}$ & $56781234_{16}$ \\
  \end{tabular}
\end{center}

\subsection{Virtual Machine Bytecode}

\begin{center}
  \begin{tabular}{ | p{6cm} | c | p{8cm} | }
    \hline
    Operation & Code & Brief Description \\ \hline
    No Operation & $00_{16}$ & Only increament the bytecode pointer by one.
                              Nothing else is done. \\
    Message Object & $01_{16}$ & Send a message to an object. \\
    Message Object with Result & $02_{16}$ & Send a message to the object
                                            and assigning the result to a
                                            local variable. \\
    Message Object's Super Class & $03_{16}$ & Nothing is \\
    Message Object's Super Class with Result & $04_{16}$ & Nothing is \\
    Beginning of Block Context & $0A_{16}$ & Nothing is \\
    End of Context & $0B_{16}$ & Nothing is \\
    Assign Local Variable & $10_{16}$ & ...\\
    Assign Object Property & $11_{16}$ & ...\\
    Assign Class Property & $12_{16}$ & ...\\
    Assign Class to Local Variable & $13_{16}$ & ...\\
    Assign an Integer Constant to a Local Variable & $14_{16}$ & ...\\
    \hline
  \end{tabular}
\end{center}

\subsection{Library File Format}

\begin{center}
  \begin{tabular}{ |l|l|p{8cm}| }
    \hline
    \multirow{5}{*}{Header} & \texttt{unsigned bytes [4]} = ''MLIB'' & The magic file identifier\\ \cline{2-3}
    & \texttt{unsigned byte} = 1 & Major Meat library format version\\ \cline{2-3}
    & \texttt{unsigned byte} = 0 & Minor Meat library format version\\ \cline{2-3}
    & \texttt{unsigned word} = 0 & Reserved for future flags.\\ \cline{2-3}
    & \texttt{unsigned byte} & The index, starting from 0, of the serialized application class. This index should never be greater than the number of serialized classes in the library unless it's the value $\mathrm{FF_{16}}$ indicating the index is not pointing to a class.\par
    If the library is being imported as a library, then this field can be ignored. Otherwise, when it is loaded by the virtual machine to be executed as an application, the index must point to a serialized application class with the appropriate overridden entry class method. When these requirements are not met, the virtual machine must raise an exception to indicate the error condition.\\ \hline

    \multirow{2}{*}{Library Imports} & \texttt{unsigned byte} & The total number of libraries that need to be imported.\\ \cline{2-3}
    & \texttt{unsigned byte arrays} & A list of ''0'' terminated strings of each library name to be imported.\\ \hline

    \multirow{2}{*}{Serialized Class Objects} & \texttt{unsigned byte} & The total number of Classes serialized in the library.\\ \cline{2-3}
    & \texttt{serialized class array} & An array of all the serialized classes. see \hyperref[sec:class_serial]{Class Serialization}.\\ \hline
  \end{tabular}
\end{center}

\subsubsection{Class Serialization}
\label{sec:class_serial}

\textbf{Serialized Class Format}
\begin{center}
  \begin{tabular}{|l|l|p{8cm}|}
    \hline
    \multirow{3}{*}{Header} & \texttt{unsigned integer} & The hashed ID of the class name.\\ \cline{2-3}
    & \texttt{unsigned byte} & The number of class properties.\\ \cline{2-3}
    & \texttt{unsigned byte} & The number of object properties.\\ \hline

    \multirow{2}{*}{Virtual Table} & \texttt{unsigned byte} & The number of virtual table entries.\\ \cline{2-3}
    & \texttt{unsigned byte} & The number of class virtual table entries.\\ \hline

    \multirow{2}{*}{Byte Code} & \texttt{unsigned word} & Size of the byte code array.\\ \cline{2-3}
    & \texttt{unsigned byte array} & The byte code.\\ \hline
  \end{tabular}
\end{center}

\textbf{Virtual Table Entries}
\begin{center}
  \begin{tabular}{|l|p{8cm}|}
    \hline
    \texttt{unsigned integer} & Hashed ID of the method name.\\ \hline
    \texttt{unsigned integer} & Hashed ID of the class where the method is found.\\ \hline
    \texttt{unsigned byte} & Entry flags.\\ \hline
    \texttt{unsigned byte} & The number of local variables required during execution.\\ \hline
    \texttt{unsigned word} & Byte code offset to the start of the method.\\ \hline
  \end{tabular}
\end{center}

\textbf{Virtual Table Entry Flags}
\begin{displaymath}
  \xymatrix @-1pc {Bit &*+[F]{7} \ar @{-} [ddd]&*+[F]{6} \ar @{-} [dd] &*+[F]{5} \ar @{-} [dd] &*+[F]{4} \ar @{-} [dd] &*+[F]{3} \ar @{-} [dd] &*+[F]{2} \ar @{-} [dd]
    &*+[F]{1} \ar @{-} [dd] &*+[F]{0} \ar @{-} [d] & \\
    & & & & & & & & \ar [r] & \txt<6cm>{If set the virtual table entry is an offset into the byte code, otherwise it is an internal function pointer.}\\
    & & \ar [rrrrrrr] & & & & & & & \txt<6cm>{Reserved for future use.}\\
    & \ar [rrrrrrrr] & & & & & & & & \txt<6cm>{Method found in a super class.}}
\end{displaymath}
